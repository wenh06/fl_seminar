%%%%%%%%%%%%%%%%%%%%%%%%%%%%%%%%%%%%%%%%%
% Beamer Presentation
% LaTeX Template
% Version 1.0 (10/11/12)
%
% This template has been downloaded from:
% http://www.LaTeXTemplates.com
%
% License:
% CC BY-NC-SA 3.0 (http://creativecommons.org/licenses/by-nc-sa/3.0/)
%
%%%%%%%%%%%%%%%%%%%%%%%%%%%%%%%%%%%%%%%%%


%%%%%%%%%%%%%%%%%%%%%%%%%%%%%%%%%%%%%%%%%

% this file contains common preambles for the slides (beamer)

% TODO: for different fontsizes, some width/height might should be adjusted

%%%%%%%%%%%%%%%%%%%%%%%%%%%%%%%%%%%%%%%%%

%----------------------------------------------------------------------------------------
%	PACKAGES AND THEMES
%----------------------------------------------------------------------------------------

\documentclass[UTF8, 12pt]{beamer}

\mode<presentation> {

% The Beamer class comes with a number of default slide themes
% which change the colors and layouts of slides. Below this is a list
% of all the themes, uncomment each in turn to see what they look like.

% for an overview of all themes and colors, ref.
% https://deic-web.uab.cat/~iblanes/beamer_gallery/index.html

%\usetheme{default}
%\usetheme{AnnArbor}
%\usetheme{Antibes}
%\usetheme{Bergen}
\usetheme[width=2\baselineskip]{Berkeley} % upper left corner
% \usetheme{Berlin}
%\usetheme{Boadilla} % vacant upper space
%\usetheme{CambridgeUS} %red
%\usetheme{Copenhagen} %pseudo-flat, round
%\usetheme{Darmstadt}
%\usetheme{Dresden}
%\usetheme{Frankfurt}
%\usetheme{Goettingen}
%\usetheme{Hannover} %left column
%\usetheme{Ilmenau}
%\usetheme{JuanLesPins}
%\usetheme{Luebeck} %flat, square
%\usetheme{Madrid}
%\usetheme{Malmoe} %brief, triangle
%\usetheme{Marburg} %right column
%\usetheme{Montpellier} %brief, upper column
%\usetheme{PaloAlto} %right column
%\usetheme{Pittsburgh}
%\usetheme{Rochester}
%\usetheme{Singapore}
%\usetheme{Szeged}
%\usetheme{Warsaw} %upper column

% As well as themes, the Beamer class has a number of color themes
% for any slide theme. Uncomment each of these in turn to see how it
% changes the colors of your current slide theme.

%\usecolortheme{albatross}
%\usecolortheme{beaver}
%\usecolortheme{beetle}
%\usecolortheme{crane}
%\usecolortheme{dolphin}
%\usecolortheme{dove}
%\usecolortheme{fly}
%\usecolortheme{lily}
%\usecolortheme{orchid}
%\usecolortheme{rose}
%\usecolortheme{seagull}
%\usecolortheme{seahorse}
%\usecolortheme{whale}
%\usecolortheme{wolverine}

\definecolor{tsinghua}{rgb}{.434,.09,.531}
\definecolor{pale}{rgb}{.9,.9,.9}
% \usecolortheme[named=tsinghua]{structure}
%\usecolortheme[named=cyan]{structure}
%\usecolortheme[named=violet]{structure}


%%%%%%%%%%%%%%%%%%%%%%%%%%%%%%%%%%%%%%%%%%%%%%%%%%%%
%
% custom color settings
%
%%%%%%%%%%%%%%%%%%%%%%%%%%%%%%%%%%%%%%%%%%%%%%%%%%%%


\definecolor{jdhblue}{RGB}{2,93,186}

\definecolor{zkblue}{RGB}{88,135,175}
\definecolor{zkbackground}{RGB}{230,232,234}


\usecolortheme[named=zkblue]{structure}

\setbeamercolor{section in head/foot}{fg=white}
% \setbeamercolor{subsection in toc}{fg=tsinghua!35}
% \setbeamercolor{frametitle}{fg=tsinghua,bg=tsinghua!20}
% \setbeamercolor{block body}{bg=tsinghua!5,fg=tsinghua}
% \setbeamercolor{author}{fg=tsinghua}
% \setbeamercolor{institute}{fg=tsinghua}
% \setbeamercolor{date}{fg=tsinghua}
% \setbeamercolor{normal text}{fg=tsinghua}
\setbeamercolor{subsection in toc}{fg=zkblue!35}
\setbeamercolor{frametitle}{fg=white,bg=zkblue}
\setbeamercolor{block body}{bg=zkblue!25,fg=zkblue}
\setbeamercolor{author}{fg=zkblue}
\setbeamercolor{institute}{fg=zkblue}
\setbeamercolor{date}{fg=zkblue}
\setbeamercolor{normal text}{fg=zkblue}
\setbeamercolor{background canvas}{bg=zkbackground}

\usefonttheme[onlymath]{serif}

%\setbeamertemplate{footline} % To remove the footer line in all slides uncomment this line
%\setbeamertemplate{footline}[page number] % To replace the footer line in all slides with a simple slide count uncomment this line

%\setbeamertemplate{navigation symbols}{} % To remove the navigation symbols from the bottom of all slides uncomment this line
}
\makeatletter
\beamer@headheight=2\baselineskip
\makeatother



%----------------------------------------------
% watermark

% \setbeamertemplate{background}{\tikz[overlay,remember picture]\node[opacity=0.4]at (current page.center){\includegraphics[width=0.8\textwidth, angle=45]{images/logo_buaa_math.jpg}};}


%----------------------------------------------


%%%%%%%%%%%%%%%%%%%%%%%%%%%%%%%%%%%%%%%%%%%%%%%%%%%%
%
% packages used
%
%%%%%%%%%%%%%%%%%%%%%%%%%%%%%%%%%%%%%%%%%%%%%%%%%%%%


\usepackage{graphicx} % Allows including images
\usepackage{pst-func}
\usepackage{float}
\usepackage[justification = centering]{caption}

\usepackage{booktabs} % Allows the use of \toprule, \midrule and \bottomrule in tables

% \usepackage[UTF8]{ctex}
% \usepackage{fontspec}

\usepackage[slantfont,boldfont]{xeCJK}
\usepackage{fontspec}

% set font
\setmainfont[Path = fonts/, BoldFont = timesbd.ttf, ItalicFont= timesi.ttf]{times.ttf}
\setsansfont[Path = fonts/, BoldFont = timesbd.ttf, ItalicFont= timesi.ttf]{times.ttf}

% set chn font
% \setCJKmainfont[Path = fonts/, BoldFont = simhei.ttf]{simfang.ttf}
% \setCJKsansfont[Path = fonts/, BoldFont = simhei.ttf]{simfang.ttf}

\setCJKmainfont[Path = fonts/, BoldFont = JDLangZhengTi.ttf]{simfang.ttf}
\setCJKsansfont[Path = fonts/, BoldFont = JDLangZhengTi.ttf]{simfang.ttf}

\usepackage{latexsym,amscd,color}
\usepackage{amsmath,amsfonts,amssymb,mathrsfs}
\usepackage{enumerate,euscript}
\usepackage{amssymb,url,xspace}
\usepackage{tikz-cd}
\usepackage[all,cmtip]{xy}
\usepackage{dashbox}
\usepackage{tikz}
\usetikzlibrary{trees,arrows,decorations.pathmorphing,shapes,shapes.geometric,backgrounds,positioning,calc,tikzmark,external}
\usepackage{xcolor}
\usepackage{hyperref}
% \usepackage[cmintegrals,cmbraces]{newtxmath}
% \usepackage{ebgaramond-maths}

\usepackage{datetime2}

\usepackage[ruled,noend]{algorithm2e}
% \usepackage{tcolorbox}
% \newtcolorbox{mybox}[2]{
%     arc=0pt,
%     boxrule=#2pt,
%     colback=#1,
%     halign=center,
%     size=small,
%     left=0pt,
%     right=0pt,
% }

% \newcommand{\tikzmark}[1]{\tikz[baseline,remember picture] \coordinate (#1) {};}

% \usepackage[round]{natbib}  % bibliography package
\usepackage{bibentry}         %  full citation in the body of the text (turn off natbib if use it)
\nobibliography*                   % no bib at the end


% https://tex.stackexchange.com/questions/30720/footnote-without-a-marker
\newcommand\blfootnote[1]{%
  \begingroup
  \renewcommand\thefootnote{}\footnote{#1}%
  \addtocounter{footnote}{-1}%
  \endgroup
}
% \usepackage[hang,flushmargin]{footmisc}



%%%%%%%%%%%%%%%%%%%%%%%%%%%%%%%%%%%%%%%%%%%%%%%%%%%%
%
% custom commands
%
%%%%%%%%%%%%%%%%%%%%%%%%%%%%%%%%%%%%%%%%%%%%%%%%%%%%

% page number in the bottom navigation bar
% https://tex.stackexchange.com/questions/137022/how-to-insert-page-number-in-beamer-navigation-symbols
% \addtobeamertemplate{navigation symbols}{}{%
%     \usebeamerfont{footline}%
%     \usebeamercolor[fg]{footline}%
%     \hspace{1em}%
%     \insertframenumber/\inserttotalframenumber
% }

% page number in the left side bar
% https://tex.stackexchange.com/questions/332895/slide-number-in-the-sidebar-of-paloalto-theme-in-beamer
\addtobeamertemplate{footline}
{%
   \usebeamercolor[fg]{author in sidebar}
   \vskip-1cm\hskip10pt
   {\bfseries\color{white}\fontsize{10pt}{10pt}\selectfont \insertframenumber\,/\,\inserttotalframenumber\kern1em\vskip2pt}
}


\newcommand*\circled[1]{
    \tikz[baseline=(char.base)]{
        \node[shape=circle,draw,color=red,inner sep=1pt] (char) {#1};
    }
}

\newcommand{\fl}{\circled{FL}}


\newcommand{\fieldc}{\mathbb{C}}
\newcommand{\fieldr}{\mathbb{R}}
\newcommand{\fieldq}{\mathbb{Q}}
\newcommand{\ringz}{\mathbb{Z}}
\newcommand{\fp}[1]{\mathbb{F}_{{#1}}}
\newcommand{\vs}{vector space}
\newcommand{\sheafl}{\mathscr{L}}
\newcommand{\GL}{\operatorname{GL}}
\newcommand{\SL}{\operatorname{SL}}
\newcommand{\gl}{\operatorname{\mathfrak{g}\mathfrak{l}}}
\newcommand{\End}{\operatorname{End}}
\newcommand{\Spec}{\operatorname{Spec}}
\newcommand{\Ind}{\operatorname{Ind}}
\newcommand{\Res}{\operatorname{Res}}
\newcommand{\Hom}{\operatorname{Hom}}
\newcommand{\Irr}{\operatorname{Irr}}
\newcommand{\Inf}{\operatorname{Inf}}

\DeclareMathOperator{\spec}{Spec}
\DeclareMathOperator{\SPEC}{\mathbf{Spec}}
\DeclareMathOperator{\proj}{Proj}
\DeclareMathOperator{\PROJ}{\mathbf{Proj}}

\DeclareMathOperator*{\argmax}{arg\,max}
\DeclareMathOperator*{\argmin}{arg\,min}
\DeclareMathOperator*{\expectation}{\mathbb{E}}


\newtheorem{thm_cn}{定理}[section]
\newtheorem{prop_cn}[thm_cn]{命题}
\newtheorem{Def_cn}[thm_cn]{定义}
\newtheorem{lem_cn}[thm_cn]{引理}
\newtheorem{conj_cn}[thm_cn]{猜想}
\newtheorem{cor_cn}[thm_cn]{推论}
\newtheorem{remark_cn}[thm_cn]{注记}
\newtheorem{eg_cn}[thm_cn]{例}
\newtheorem{recall_cn}[thm_cn]{回忆}
\newtheorem{question_cn}[thm_cn]{问题}
\newtheorem{observation_cn}[thm_cn]{观察}

\newtheorem{thm}{Theorem}[section]
\newtheorem{prop}[thm]{Proposition}
\newtheorem{Def}[thm]{Definition}
\newtheorem{lem}[thm]{Lemma}
\newtheorem{conj}[thm]{Conjecture}
\newtheorem{cor}[thm]{Corollary}
\newtheorem{remark}[thm]{Remark}
\newtheorem{eg}[thm]{Example}
\newtheorem{recall}[thm]{Recall}
\newtheorem{question}[thm]{Question}
\newtheorem{observation}[thm]{Observation}


\setlength\parindent{1em}
\addtobeamertemplate{frametitle}{\setlength{\parindent}{0em}}{}
\addtobeamertemplate{block begin}{\setlength{\parindent}{0em}}{\setlength\parindent{1em}}

\newif\ifshowtoc
\showtoctrue % toggles to show the toc

\AtBeginSection{%
\ifshowtoc
\begin{frame}
    \tableofcontents[currentsection, subsectionstyle=show/hide/hide]
\end{frame}
\fi
}

\logo{\includegraphics[height=1.5cm]{images/logo_buaa_math.jpg}}

% \makeatletter
% \newcommand\insertlogoii{}
% \newcommand\logoii[1]{\renewcommand\insertlogoii{#1}}
% \setbeamertemplate{headline}
%   {%
%     \begin{beamercolorbox}[wd=\paperwidth]{frametitle}
%       \ifx\beamer@sidebarside\beamer@lefttext%
%       \else%
%         \hfill%
%       \fi%
%       \ifdim\beamer@sidebarwidth>0pt%  
%         \usebeamercolor[bg]{logo}%
%         \vrule width\beamer@sidebarwidth height \beamer@headheight%
%         \hskip-\beamer@sidebarwidth%
%         \hbox to \beamer@sidebarwidth{\hss\vbox to
%           \beamer@headheight{\vss\hbox{\color{fg}\insertlogo}\vss}\hss}%
%         \hfill%
%         \vrule width\beamer@sidebarwidth height \beamer@headheight%
%         \hskip-\beamer@sidebarwidth%
%         \hbox to \beamer@sidebarwidth{\hss\vbox to
%           \beamer@headheight{\vss\hbox{\color{fg}\insertlogoii}\vss}\hss}%
%       \else%
%         \vrule width0pt height \beamer@headheight%  
%       \fi%
%     \end{beamercolorbox}
% }
% \makeatother


\allowdisplaybreaks

\setbeamertemplate{bibliography item}{\insertbiblabel}
