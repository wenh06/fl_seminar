%%%%%%%%%%%%%%%%%%%%%%%%%%%%%%%%%%%%%%%%%
% Beamer Presentation
% LaTeX Template
% Version 1.0 (10/11/12)
%
% This template has been downloaded from:
% http://www.LaTeXTemplates.com
%
% License:
% CC BY-NC-SA 3.0 (http://creativecommons.org/licenses/by-nc-sa/3.0/)
%
%%%%%%%%%%%%%%%%%%%%%%%%%%%%%%%%%%%%%%%%%

%%%%%%%%%%%%%%%%%%%%%%%%%%%%%%%%%%%%%%%%%

% this file contains common preambles for the notes/articles, etc. (other than beamer)

%%%%%%%%%%%%%%%%%%%%%%%%%%%%%%%%%%%%%%%%%

%----------------------------------------------------------------------------------------
%	PACKAGES AND THEMES
%----------------------------------------------------------------------------------------

\documentclass[UTF8, 12pt]{article}

\usepackage{graphicx} % Allows including images
\usepackage{pst-func}
\usepackage{float}
\usepackage[justification = centering]{caption}

\usepackage{booktabs} % Allows the use of \toprule, \midrule and \bottomrule in tables

% \usepackage[UTF8]{ctex}
% \usepackage{fontspec}

\usepackage{hyperref}
\hypersetup{
    colorlinks=true,
    linkcolor=blue,
    filecolor=magenta,      
    urlcolor=cyan,
}

\usepackage[slantfont,boldfont]{xeCJK}
\usepackage{fontspec}

% set font
\setmainfont[Path = fonts/, BoldFont = timesbd.ttf, ItalicFont= timesi.ttf]{times.ttf}
\setsansfont[Path = fonts/, BoldFont = timesbd.ttf, ItalicFont= timesi.ttf]{times.ttf}

% set chn font
% \setCJKmainfont[Path = fonts/, BoldFont = simhei.ttf]{simfang.ttf}
% \setCJKsansfont[Path = fonts/, BoldFont = simhei.ttf]{simfang.ttf}

\setCJKmainfont[Path = fonts/, BoldFont = JDLangZhengTi.ttf]{simfang.ttf}
\setCJKsansfont[Path = fonts/, BoldFont = JDLangZhengTi.ttf]{simfang.ttf}

\usepackage{latexsym,amscd,color}
\usepackage{amsmath,amsfonts,amssymb,mathrsfs}
\usepackage{enumerate,euscript}
\usepackage{url,xspace}
\usepackage{tikz-cd}
\usepackage[all,cmtip]{xy}
\usepackage{dashbox}
\usepackage{tikz}
\usetikzlibrary{trees,arrows,decorations.pathmorphing,shapes,backgrounds,positioning,calc,tikzmark}
\usepackage{xcolor}
\usepackage{hyperref}
% \usepackage[cmintegrals,cmbraces]{newtxmath}
% \usepackage{ebgaramond-maths}

\usepackage{datetime2}

\usepackage[ruled,noend]{algorithm2e}
% \usepackage{tcolorbox}
% \newtcolorbox{mybox}[2]{
%     arc=0pt,
%     boxrule=#2pt,
%     colback=#1,
%     halign=center,
% }

% \newcommand{\tikzmark}[1]{\tikz[baseline,remember picture] \coordinate (#1) {};}

% \usepackage{natbib}  % bibliography package
\usepackage{bibentry}         %  full citation in the body of the text (turn off natbib if use it)
\nobibliography*                   % no bib at the end


% https://tex.stackexchange.com/questions/30720/footnote-without-a-marker
\newcommand\blfootnote[1]{%
  \begingroup
  \renewcommand\thefootnote{}\footnote{#1}%
  \addtocounter{footnote}{-1}%
  \endgroup
}
% \usepackage[hang,flushmargin]{footmisc}



\newcommand*\circled[1]{\tikz[baseline=(char.base)]{ \node[shape=circle,draw,inner sep=2pt] (char) {#1};}}


\newcommand{\fieldc}{\mathbb{C}}
\newcommand{\fieldr}{\mathbb{R}}
\newcommand{\fieldq}{\mathbb{Q}}
\newcommand{\ringz}{\mathbb{Z}}
\newcommand{\fp}[1]{\mathbb{F}_{{#1}}}
\newcommand{\vs}{vector space}
\newcommand{\sheafl}{\mathscr{L}}
\newcommand{\GL}{\operatorname{GL}}
\newcommand{\SL}{\operatorname{SL}}
\newcommand{\gl}{\operatorname{\mathfrak{g}\mathfrak{l}}}
\newcommand{\End}{\operatorname{End}}
\newcommand{\Spec}{\operatorname{Spec}}
\newcommand{\Ind}{\operatorname{Ind}}
\newcommand{\Res}{\operatorname{Res}}
\newcommand{\Hom}{\operatorname{Hom}}
\newcommand{\Irr}{\operatorname{Irr}}
\newcommand{\Inf}{\operatorname{Inf}}

\DeclareMathOperator{\spec}{Spec}
\DeclareMathOperator{\SPEC}{\mathbf{Spec}}
\DeclareMathOperator{\proj}{Proj}
\DeclareMathOperator{\PROJ}{\mathbf{Proj}}

\DeclareMathOperator*{\argmax}{arg\,max}
\DeclareMathOperator*{\argmin}{arg\,min}
\DeclareMathOperator*{\expectation}{\mathbb{E}}


\newtheorem{thm_cn}{定理}[section]
\newtheorem{prop_cn}[thm_cn]{命题}
\newtheorem{Def_cn}[thm_cn]{定义}
\newtheorem{lem_cn}[thm_cn]{引理}
\newtheorem{conj_cn}[thm_cn]{猜想}
\newtheorem{cor_cn}[thm_cn]{推论}
\newtheorem{remark_cn}[thm_cn]{注记}
\newtheorem{eg_cn}[thm_cn]{例}
\newtheorem{recall_cn}[thm_cn]{回忆}
\newtheorem{question_cn}[thm_cn]{问题}
\newtheorem{observation_cn}[thm_cn]{观察}

\newtheorem{thm}{Theorem}[section]
\newtheorem{prop}[thm]{Proposition}
\newtheorem{Def}[thm]{Definition}
\newtheorem{lem}[thm]{Lemma}
\newtheorem{conj}[thm]{Conjecture}
\newtheorem{cor}[thm]{Corollary}
\newtheorem{remark}[thm]{Remark}
\newtheorem{eg}[thm]{Example}
\newtheorem{recall}[thm]{Recall}
\newtheorem{question}[thm]{Question}
\newtheorem{observation}[thm]{Observation}

\allowdisplaybreaks

